This package implements a Robot\+Driver and several applications for the Tri\+Finger robots, using the \href{https://open-dynamic-robot-initiative.github.io/code_documentation/robot_interfaces/docs/doxygen/html/index.html}{\tt {\ttfamily robot\+\_\+interfaces} package}. See there for more information on the general architecture. The action and observation types for the (Tri)Finger robots are also implemented in {\ttfamily robot\+\_\+interfaces}.

We provide several \href{https://github.com/open-dynamic-robot-initiative/robot_fingers/blob/master/demos}{\tt demos} to show how to use the interface on practical examples. Good starting points are\+:


\begin{DoxyItemize}
\item \href{https://github.com/open-dynamic-robot-initiative/robot_fingers/blob/master/demos/demo_real_finger.py}{\tt demo\+\_\+real\+\_\+finger.\+py}\+: Basic example on how to control the robot using either torque or position commands. This uses only a single finger but the principle is the same for the Tri\+Finger.
\item \href{https://github.com/open-dynamic-robot-initiative/robot_fingers/blob/master/demos/demo_trifinger.py}{\tt demo\+\_\+trifinger.\+py}\+: Demo for the Tri\+Finger, moving it in a hard-\/coded choreography.
\end{DoxyItemize}

\begin{DoxyNote}{Note}
The demos are all in Python, however, you can do exactly the same using the C++ A\+PI. 
\end{DoxyNote}
