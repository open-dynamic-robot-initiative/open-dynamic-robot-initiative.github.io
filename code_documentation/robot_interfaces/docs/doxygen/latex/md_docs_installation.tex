\begin{DoxyNote}{Note}
If you intend to use this interface to control your own robot, this package (and its dependencies) is enough, and you can follow the instructions below. If you are looking for the interface of the Tri\+Finger robot interface, see the installation instructions of the \href{https://open-dynamic-robot-initiative.github.io/code_documentation/robot_fingers/docs/doxygen/html/index.html}{\tt {\ttfamily robot\+\_\+fingers} package} instead (this also includes {\ttfamily robot\+\_\+interfaces}).
\end{DoxyNote}
\subsection*{Dependencies }

We are using \href{http://wiki.ros.org/catkin}{\tt catkin} as build tool (i.\+e. {\ttfamily robot\+\_\+interfaces} is a catkin package). While we are not really depending on any \href{http://www.ros.org}{\tt R\+OS} packages, this means you need a basic R\+OS installation to build.

In the following we are using catkin\+\_\+tools which need to be installed separately\+: \begin{DoxyVerb}pip install catkin_tools
\end{DoxyVerb}


We are testing on Ubuntu 18.\+04 with R\+OS Melodic. Other versions may work as well but are not officially supported.

\begin{DoxyNote}{Note}
We provide a Singularity image with all dependencies for the Tri\+Finger robot which also covers everything needed for {\ttfamily robot\+\_\+interfaces}. See the documentation of the \href{https://open-dynamic-robot-initiative.github.io/code_documentation/robot_fingers/docs/doxygen/html/index.html}{\tt {\ttfamily robot\+\_\+fingers} package} for more information.
\end{DoxyNote}
\subsection*{Get the Source }

{\ttfamily robot\+\_\+interfaces} depends on several other of our packages which are organized in separate repositories. We therefore use a workspace management tool called \href{https://pypi.org/project/treep/}{\tt treep} which allows easy cloning of multi-\/repository projects.

treep can be installed via pip\+: \begin{DoxyVerb}pip install treep
\end{DoxyVerb}


Clone the treep configuration containing the \char`\"{}\+R\+O\+B\+O\+T\+\_\+\+I\+N\+T\+E\+R\+F\+A\+C\+E\+S\char`\"{} project\+: \begin{DoxyVerb}git clone git@github.com:machines-in-motion/treep_machines_in_motion.git
\end{DoxyVerb}


\begin{DoxyNote}{Note}
treep searches for a configuration directory from the current working directory upwards. So you can use treep in the directory in which you invoked the {\ttfamily git clone} command above or any subdirectory.
\end{DoxyNote}
Now clone the project\+: \begin{DoxyVerb}treep --clone ROBOT_INTERFACES
\end{DoxyVerb}


\begin{DoxyNote}{Note}
{\bfseries Important\+:} treep uses S\+SH to clone from github. So for the above command to work, you need a github account with a registered S\+SH key. Further this key needs to work without asking for a password everytime. To achieve this, run \begin{DoxyVerb}ssh-add
\end{DoxyVerb}

\end{DoxyNote}
first.

You should now have the following directory structure\+: \begin{DoxyVerb}├── treep_machines_in_motion
└── workspace
    └── src
        ├── catkin
        │   ├── core_robotics
        │   │   ├── mpi_cmake_modules
        │   │   ├── pybind11_catkin
        │   │   ├── real_time_tools
        │   │   ├── shared_memory
        │   │   ├── time_series
        │   │   └── yaml_cpp_catkin
        │   ├── examples
        │   │   └── ci_example
        │   ├── robots
        │   │   └── robot_interfaces
        │   └── tools
        │       ├── serialization_utils
        │       └── signal_handler
        └── not_catkin
            └── third_party
                └── pybind11
\end{DoxyVerb}


\subsection*{Build }

To build, cd into the {\ttfamily workspace} directory and call \begin{DoxyVerb}catkin build
\end{DoxyVerb}


to build the whole workspace.

\subsubsection*{Python Bindings}

With the above command Python bindings will be build for the default python version of your system (see {\ttfamily python -\/-\/version}). If you want to use a different version (e.\+g. python3), you can specify as follows\+: \begin{DoxyVerb}catkin build -DPYTHON_EXECUTABLE=/usr/bin/python3\end{DoxyVerb}
 